\section{Einführung (4P)}

\subsection{Übersicht über die Applikation (1P)}
\task{Was macht die Applikation? Wie funktioniert sie? Welches Problem löst sie/welchen Zweck hat sie?}
Bilanzius ist ein Finanzsystem, welches die Finanzen Mehrer Benutzter verwalten soll. Dazu zählen das Verwalten von mehreren Konten, Geld sowie Kategorien. 

\subsection{Starten der Applikation (1P)}
\task{Wie startet man die Applikation? Was für Voraussetzungen werden benötigt? Schritt-für-Schritt-Anleitung}
Gebaute App:  
\begin{itemize}
    \item Im Terminal \textit{java –jar bilanzius.jar} ausführen
\end{itemize}
Code: 
\begin{itemize}
    \item Projekt in preferierter IDE öffnen
    \item Datei org.bilanzius.Main öffnen
    \item main() Funktion über die IDE starten
\end{itemize}

\subsection{Technischer Überblick (2P)}
\task{Nennung und Erläuterung der Technologien (z.B. Java, MySQL, …), jeweils Begründung für den Einsatz der Technologien}
\begin{itemize}
    \item Java – In der Vorlesung verwendete Programmiersprache.
    \item SQLite – Lokale Datenbank, die keine extra Installation benötigt.
    \item Maven – Dependency Manager sowie Build Tool
    \item GSON – Bibliothek, um JSON-Daten zu verwalten
    \item JUnit - Als Framework für automatisierte Softwaretests
\end{itemize}